% title page
\pagenumbering{Alph}
\begin{center}
% {\normalsize MODEL ESTIMATION AND DYNAMIC PREDICTIONS FOR SUBJECT-SPECIFIC \par EVENT PROBABILITIES IN JOINT MODEL USING \par LONGITUDINAL QUANTILE REGRESSION}
{\normalsize BAYESIAN QUANTILE REGRESSION JOINT\par
MODELS: INFERENCE AND\par
DYNAMIC PREDICTIONS}
\end{center}\par

% \ \par
{\normalsize
\begin{center}
\singlespacing
by\\
\ \par
MING YANG, MS\par
\ \par
APPROVED:\par
\ \par
\ \par
\end{center}}

% \doublespacing
\begin{table}[h]
\begin{flushright}
	\begin{tabular}{ p{8cm}}

	\hline
	STACIA DESANTIS, PhD\\
	\\
	\\
	\\
	\hline
	SHENG LUO, PhD\\
	\\
	\\
	\\
	\hline
	DAVID LAIRSON, PhD\\
	\\
	\\
	\\
	\hline
	XIAOMING LIU, PhD\\
	\\
	\\
	\\
	\hline
	DEAN, THE UNIVERSITY OF TEXAS\par
	SCHOOL OF PUBLIC HEALTH\\[0.8cm]
\end{tabular}
\end{flushright}
\label{default}
\end{table}

\thispagestyle{empty}
% \pagestyle{empty}
% \end{titlepage}


%% copyright page
\newpage
\thispagestyle{empty}
\begin{center}
\singlespacing
Coptyright\\
by\\
Ming Yang, MS, PhD\\
2016
\end{center}


%% dedication page
\newpage
\thispagestyle{empty}
\doublespacing
\begin{center}
DEDICATION\\
To my family.\\
\end{center}


%% thesis information and author information page
\newpage
\thispagestyle{empty}
\doublespacing
\begin{center}

{\normalsize  BAYESIAN QUANTILE REGRESSION JOINT\par
MODELS: INFERENCE AND\par
DYNAMIC PREDICTIONS}

\ \par
\ \par
by\par
\singlespacing
MING YANG, \\
BS, Xiamen University, 2008\\
MS, The University of Texas School of Public Health, 2012\par

\ \par
\doublespacing
Presented to the Faculty of The University of Texas\\
School of Public Health\\
in Partial Fulfillment\\
of the Requirements\\
for the Degree of\par
\ \par
DOCTOR OF PHILOSOPHY\par
\ \par
\ \par
\singlespacing
THE UNIVERSITY OF TEXAS\\
SCHOOL OF PULIC HEALTH\\
Houston, Texas\\
December, 2016
\end{center}


%%% ACKNOWLEDGEMENTS page
\newpage
\thispagestyle{empty}
\doublespacing
\begin{center}
ACKNOWLEDGEMENTS
\end{center}
I would like to gratefully and sincerely thank my dissertation co-advisors Dr. Stacia DeSantis and Dr. Sheng Luo. It would not have been possible to write this dissertation without their ideas, patience and guidance. They have devoted their precious time to serve as my advisors and I appreciate for all their helpful discussion and comments throughout our countless meetings and email communications. I would like to thank my minor advisor Dr. David Lairson and breadth advisor Dr. Xiaoming Liu for their support and inputs to my dissertation. Thanks also go to my external reviewer Dr. Soeun Kim for her consistent support and precious time towards my requests during this work. Last but not least, I would like to thank my wife Yalan, for her encouragement, quiet patience and unwavering love. Her tolerance of my occasional vulgar moods is a testament in itself of her unyielding devotion and love. 


%%% short summary
\newpage
% \thispagestyle{empty}
\pagenumbering{gobble}
\doublespacing
\begin{center}
{\normalsize BAYESIAN QUANTILE REGRESSION JOINT\par
MODELS: INFERENCE AND\par
DYNAMIC PREDICTIONS}\par
\ \par
\singlespacing
Ming Yang, PhD\\
The University of Texas\\
School of Public Health, 2016
\end{center}

\doublespacing
\noindent
Dissertation Chair, Stacia DeSantis, PhD\\

\indent
In the traditional joint models (JM) of a longitudinal and time-to-event data, a linear mixed model (LMM) assuming normal random error is frequently used to model the longitudinal continuous outcome. However, in many circumstances, the normality assumption cannot be satisfied and LMM is not appropriate to use. In addition, as a mean regression based methods, LMM only models the conditional mean of the longitudinal outcome, thus its application is limited when clinical interest lies in making inference or predictions on median, lower, or upper ends of the outcome variable. In contrast, quantile regression (QR) models provide a more flexible, distribution-free way to study covariate effects at different conditional quantiles of the outcome and it is robust against deviations from normality as well as outlying observations. In addition, the JM framework provides a convenient way to make subject-specific predictions of event probability. However, current predictive algorithms are all based on the traditional JM that uses LMM. In the first paper, we proposed a new version of JM that adopts a linear quantile mixed model (LQMM) for the longitudinal process and we named it quantile regression joint models (QRJM). We developed a Gibbs sampling algorithm based on the location-scale representation of the asymmetric Laplace distribution, assessed its performance through extensive simulation studies, and demonstrated how the QRJM approach can be used for making subject-specific dynamic predictions of the risk of Huntington's disease onset. As another type of time-to-event outcome, recurrent events are commonly encountered in longitudinal biomedical studies. In contrast to survival data, multiple event times are observed in a single subject during the study follow-up. In the second paper, we extended the proposed QRJM in paper 1 to joint analysis of longitudinal and recurrent event data and developed a fully Bayesian algorithm for model inference. In the third paper, we developed a subject-specific dynamic prediction algorithm for recurrent event probability based on the QRJM proposed in paper 2. We conducted extensive simulation studies to validate the proposed algorithm in inference (paper 2) and to quantify its predictive performance (paper 3). In the data applications of paper 2 and 3, we illustrated the flexibility of the QRJM and its advantages over the traditional JM by jointly modeling the risk of coronary heart disease (CHD) recurrences and longitudinal systolic blood pressure (SBP) measurements (paper 2) and by making predictions of the risk of CHD recurrences (paper 3). QRJM was able to provide more insight into the disease progression and the association between the two disease processes in terms of various quantile-based estimations and dynamic predictions.

% In conclusion, our proposed JM that uses LQMM for longitudinal data is an important supplement to the traditional JM. It is more robust against the deviation from the normality assumption of the random error and has the flexibility of studying the covariates effects on different quantiles of the outcome. Moreover, the idea subject-specific dynamic prediction fits into the big concept of ``precision medicine'' and plays an important role in the transition from traditional population average health care to personalized medicine.
