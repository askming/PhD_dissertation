% \begin{center}
\title{Bayesian Quantile Regression Joint Models: Dynamic Predictions of Recurrent Event Probability}
% \end{center}

\author{}
\date{}

\maketitle


\begin{center}
{\bf Abstract}
\end{center}

Joint models (JM) of longitudinal and time-to-event outcomes have received increasing interest recently. A novel use of JM is to make dynamic predictions of event probability from observed longitudinal and time-to-event data. In contrast to the extensive literature on JM of longitudinal and single time-to-event (e.g. death) data, less attention has been received for the JM of longitudinal and recurrent event data. In this work, we develop a Gibbs sampling algorithm for making subject-specific dynamic predictions of the risk of event recurrence based the JM of longitudinal and recurrent event outcomes. In our JM, differently from traditional JM, we adopt a linear quantile mixed model (LQMM) instead of the frequently used linear mixed model (LMM) for the longitudinal outcome. Compared with LMM, as a quantile regression based model, LQMM is more robust against non-normality or outliers in the data. Moreover, LQMM is more flexible than LMM in that it allows investigation of covariate effects on any conditional quantile of the outcome. In the proposed Bayesian algorithm, predictions are calculated based on the entire longitudinal trajectory as well as the recurrent events history and can be dynamically updated when new data from either (or both) outcome is available. In addition, implemented through the MCMC technique, the uncertainty of the predictive inference is fully captured in the posterior distribution and no asymptotic theory is needed to derive the standard error. It is straightforward to code and implement the proposed Bayesian predictive algorithm in existing software. We assess the performance of our model through extensive simulation studies and apply it to dynamically predict the probability of recurrent coronary heart diseases for the Atherosclerosis Risk In Communities Study cohort.

% Recurrent event outcome (e.g. multiple heart failures, cancer recurrence, hospital readmissions, etc.)  is commonly encountered in longitudinal biomedical studies. In those studies, recurrent events can be an important outcome of interest in monitoring disease progression or health condition of the study participants in addition to some continuous longitudinal disease biomarkers. For example, in the Atherosclerosis Risk In Communities Study (ARIC), recurrent event outcomes, including coronary heart disease, stroke, and heart failure, etc., were collected from extensive cohort surveillance to monitor the health condition of the study cohort; meanwhile, repeated systolic blood pressure (SBP) were also measured at regular clinical visits. Previous studies using ARIC data found positive correlation between SBP and recurrent CHD events in the study cohort and patients who had CHD event(s) before were at higher risk to have disease recurrence in the future. This data mechanism motivates us to consider the joint models (JM) of longitudinal and recurrent event data for modeling the two outcomes together and furthermore to make predictions of future event probability based on historical data. In this paper, we propose to use the so-called linear quantile mixed model (LQMM) for the longitudinal outcome jointly with a proportional hazard model (PHM) for the recurrent event outcome under the JM framework. In contrast to the traditional JM methodology, where a linear mixed model (LMM) is commonly used for modeling the conditional mean of the longitudinal outcome, our quantile regression (QR) based model provides a more flexible, distribution-free way to study covariate effects at different quantiles of the longitudinal outcome. This becomes extremely important when higher (e.g. SBP) or lower (e.g. CD4 count) tail of the outcome is more relevant to clinical interest. Based on the proposed JM, we develop a novel dynamic prediction algorithm for predicting the probability of disease recurrence in the future. In this work, we develop a Gibbs sampling algorithm for the model inference as well as for dynamic predictions. The proposed Bayesian algorithm is based on the location-scale representation of the asymmetric Laplace distribution for LQMM and implemented through the MCMC technique. By using fully Bayesian algorithm the parameter uncertainly is conveniently incorporated into the predictions. We assess the performance of our Bayesian algorithms through extensive simulation studies and apply the proposed JM to dynamically predict the probability of recurrent CHD event for the ARIC cohort.



% Recurrent event data are commonly encountered in longitudinal follow-up studies related to biomedical science, econometrics, reliability, and demography. In many studies, recurrent events serve as important measurements for evaluating disease progression, health deterioration, or insurance risk. When analyzing recurrent event data, an independent censoring condition is typically required for the construction of statistical methods. In some situations, however, the terminating time for observing recurrent events could be correlated with the recurrent event process, thus violating the assumption of independent censoring. In this article, we consider joint modeling of a recurrent event process and a failure time in which a common subject-specific latent variable is used to model the association between the intensity of the recurrent event process and the hazard of the failure time. The proposed joint model is flexible in that no parametric assumptions on the distributions of censoring times and latent variables are made, and under the model, informative censoring is allowed for observing both the recurrent events and failure times. We propose a "borrow-strength estimation procedure" by first estimating the value of the latent variable from recurrent event data, then using the estimated value in the failure time model. Some interesting implications and trajectories of the proposed model are presented. Properties of the regression parameter estimates and the estimated baseline cumulative hazard functions are also studied.

{\bf Key words:} Bayesian; Dynamic predictions; Joint models; Linear quantile mixed model; Recurrent events.