% \documentclass[12pt]{article}
% \usepackage[text={7.1 in, 8 in}, top=1.75in, left=0.69in]{geometry}
% \usepackage[round]{natbib}
% \usepackage[utf8]{inputenc}
% \usepackage{fullpage}
% \usepackage {setspace}
% \usepackage[hang,flushmargin]{footmisc} %control footnote indent
% \usepackage{url} % for website links
% \usepackage{amssymb,amsmath}%for matrix
% \usepackage{graphicx}%for figure
% \usepackage{appendix}%for appendix
% \usepackage{float}
% \floatstyle{plaintop}
% \restylefloat{table}
% \usepackage{multirow}
% \usepackage{longtable}
% \usepackage{morefloats}%in case there are too many float tables and figures
% \usepackage{caption}
% % \usepackage{subcaption}
% \usepackage{subfig}
% \captionsetup[subtable]{font=normal}
% \usepackage{hyperref}
% \usepackage{courier}
% \usepackage{color}
% \usepackage{setspace}
% \usepackage{rotating} % rotate table by some degree
% \usepackage{rotfloat}
% \usepackage{mwe}
% \usepackage{listings}
% \usepackage{titling}
% \usepackage{lipsum}
% \usepackage[export]{adjustbox}




% \graphicspath{{figures/}}

% \begin{document}
% \author{\small{\bf Ming Yang, Sheng Luo,
% and Stacia M. DeSantis}\\
% \small{Department of Biostatistics, The University of Texas Health Science Center at Houston}}
% \date{}
% \title{\large{\bf Web-based Supplementary Material for ``Model Estimation and Dynamic Prediction in Joint Modeling Using Longitudinal Quantile Regression''}}
% \maketitle{}


\subsection*{Appendices}
\addcontentsline{toc}{subsection}{Appendices}
\renewcommand{\thesubsubsection}{\Alph{subsubsection}}


\subsubsection{Additional Simulation Results}\label{sec:p2appendix_simulation}
\begin{table}[H]
\centering
\caption{Simulation result for Scenario 3 in which random errors are generated from ALD($0, 1, \tau$ = 0.50).}
\label{tab:p2simsce2}
\begin{tabular}{clccccccccc}
\hline
& & \multicolumn{4}{c}{QRJM ($\tau=0.5$)} & & \multicolumn{4}{c}{LMJM}\\
\hline
 & & Bias &SE & MSE & CP && Bias & SE & MSE & CP \\
 \hline
\multirow{9}{*}{$n=250$} &  \multicolumn{8}{l}{Coefficients for longitudinal process}\\
  & $\beta_1$ & 0.007 & 0.087 & 0.007 & 0.970 && 0.010 & 0.094 & 0.009 & 0.970 \\
  & $\beta_2$ & $-$0.008 & 0.162 & 0.028 & 0.925 && $-$0.005 & 0.167 & 0.029 & 0.935 \\
  & $\beta_3$ & 0.026 & 0.062 & 0.004 & 0.945 && 0.032 & 0.073 & 0.006 & 0.935 \\
  & $\sigma$ & $-$0.002 & 0.031 & 0.001 & 0.935 &&  $-$ & $-$ & $-$ & $-$ \\
  &  \multicolumn{8}{l}{Coefficients for recurrent event process}\\
  & $\gamma$ & 0.006 & 0.070 & 0.005 & 0.960 && 0.005 & 0.075 & 0.006 & 0.930 \\
  & $r_0$ & 0.019 & 0.124 & 0.016 & 0.950 && 0.026 & 0.142 & 0.018 & 0.965 \\
  & $\alpha$ & 0.016 & 0.065& 0.004 & 0.940 && 0.015 &0.073 & 0.006 & 0.950 \\
   \hline\hline
  \multirow{9}{*}{$n=500$} & \multicolumn{8}{l}{Coefficients for longitudinal process}\\
  & $\beta_1$ & $-$0.001 & 0.061 & 0.004 & 0.920 && $-$0.000 & 0.066 & 0.005 & 0.925 \\
  & $\beta_2$ & $-$0.002 & 0.113 & 0.011 & 0.965 && $-$0.001 & 0.117 & 0.011 & 0.965 \\
  & $\beta_3$ & 0.015 & 0.044 & 0.002 & 0.950 & & 0.018 & 0.051 & 0.003 & 0.935 \\
  & $\sigma$ & 0.002 & 0.022 & 0.001 & 0.940 &&  $-$ & $-$ & $-$ & $-$ \\
  &  \multicolumn{8}{l}{Coefficients for recurrent event process}\\
  & $\gamma$ & 0.008 & 0.050 & 0.003 & 0.915 && 0.008 & 0.054 & 0.004 & 0.920 \\
  & $r_0$ & $-$0.016 & 0.087 & 0.008 & 0.940 && $-$0.018 & 0.099 & 0.009 & 0.945 \\
  & $\alpha$ & 0.003 & 0.047 & 0.003 & 0.940 && 0.003 & 0.054 & 0.003 & 0.975 \\
   \hline
\end{tabular}
\end{table}



\begin{table}[H]
\centering
\caption{Simulation result for Scenario 4 in which random errors are generated from ALD($0, 1, \tau$ = 0.75).}
\label{tab:p2simsce3}
\adjustbox{max width=\textwidth}{
\begin{tabular}{clcccccccccccccc}
\hline
& & \multicolumn{4}{c}{QRJM ($\tau=0.75$)} & & \multicolumn{4}{c}{QRJM ($\tau=0.5$)} & & \multicolumn{4}{c}{LMJM}\\
\hline
 & & Bias & SE & MSE & CP && Bias & SE & MSE & CP & & Bias & SE & MSE & CP\\
\hline
\multirow{9}{*}{$n=250$} &  \multicolumn{12}{l}{Coefficients for longitudinal process}\\
  & $\beta_1$ & 0.006 & 0.091 & 0.007 & 0.970 && 0.033 & 0.104 & 0.013 & 0.950 && 0.142 & 0.111 & 0.030 & 0.800\\
  &   $\beta_2$ &  $-$0.008 & 0.165 & 0.029 & 0.930 && 0.015 & 0.176 & 0.031 & 0.950 && 0.130 & 0.193 & 0.050 & 0.940\\
  &   $\beta_3$ & 0.028 & 0.068 & 0.005 & 0.940 && 0.048 & 0.083 & 0.010 & 0.890 && 0.146 & 0.092 & 0.029 & 0.665\\
  &   $\sigma$ & $-$0.002 & 0.031 & 0.001 & 0.940 && $-$0.323 & 0.021 & 0.105 & 0.000 && $-$ & $-$ & $-$ & $-$ \\
  &  \multicolumn{12}{l}{Coefficients for recurrent event process}\\
  &   $\gamma$ & 0.010 & 0.074 & 0.005 & 0.940 && $-$0.047 & 0.082& 0.008 & 0.930 &&  $-$0.000 & 0.079 & 0.007 & 0.925\\
  &   $r_0$ & 0.018 & 0.137 & 0.017 & 0.965 && 4.468 & 0.636 & 20.354 & 0.000 && 8.599 & 0.431 & 73.958 & 0.000\\
  &   $\alpha$ & 0.016 & 0.070& 0.004 & 0.955 && $-$0.020 & 0.083 & 0.008 & 0.930 && $-$0.133 & 0.067 & 0.022 & 0.515\\
   \hline\hline
  \multirow{9}{*}{$n=500$}&  \multicolumn{12}{l}{Coefficients for longitudinal process}\\
  & $\beta_1$ & $-$0.005 & 0.064 & 0.005 & 0.940 && 0.012 & 0.072 & 0.006 & 0.900 && 0.111 & 0.076 & 0.019 & 0.685\\
  & $\beta_2$ & $-$0.001 & 0.116 & 0.015 & 0.920 && 0.007 & 0.122 & 0.014 & 0.960 && 0.101 & 0.133 & 0.026 & 0.895\\
  & $\beta_3$ & 0.027 & 0.048 & 0.003 & 0.900 && 0.029 & 0.058 & 0.004 & 0.935 && 0.111 & 0.064 & 0.015 & 0.590\\
  & $\sigma$ & 0.003 & 0.022 & 0.001 & 0.940 && $-$0.317 & 0.015 & 0.101 & 0.000 && $-$ & $-$ & $-$ & $-$ \\
  &  \multicolumn{12}{l}{Coefficients for recurrent event process}\\
  & $\gamma$ & 0.005 & 0.052 & 0.002 & 0.955 && 0.006 & 0.056 & 0.004 & 0.940 && $-$0.004 & 0.056 & 0.004 & 0.910\\
  & $r_0$ & $-$0.003 & 0.096 & 0.010 & 0.935 && 4.377 & 0.436 & 19.469 & 0.000 && 8.773 & 0.267 & 76.972 & 0.000\\
  & $\alpha$ & 0.001 & 0.051 & 0.003 & 0.940 && $-$0.001 & 0.060 & 0.004 & 0.920 && $-$0.104 & 0.049 & 0.013 & 0.400\\
   \hline
\end{tabular}
}
\end{table}






\newpage
% \thispagestyle{lscape}
% \begin{landscape}
% \doublespacing
% \begin{sidewaystable}[H]
\subsubsection{Summary Table of Study Cohort Characteristics}
\begin{table}[H]
\centering
\caption{Baseline characteristics of study cohort with stratification by SBP level}
\label{tab:p2cht_baseline}
% \adjustbox{max width=\textwidth}{
\resizebox{\linewidth}{!}{
\begin{tabular}{llcccc}
\hline
& & \multicolumn{3}{c}{SBP groups (mm Hg)} \\
\cline{3-5}
 Characteristics\textsuperscript{$\dagger$} & Total ($n=657$) & $<$ 120 ($n=133$, 20.2\%)  & [120, 140) ($n=217$, 33.0\%)& $\ge$ 140 ($n=307$, 46.7\%) & $p$-value\textsuperscript{*}\\
 \hline
 Age & 56.4 (5.8) & 55.0 (5.7) & 55.7 (6.0) & 57.4 (5.4) & $<$0.001\\
 SBP & 135.9 (18.5) & 110.5 (6.9) & 129.2 (5.7) & 151.6 (11.2) & $<$0.001\\
 Cholesterol (mg/dL)& 215.9 (41.7) & 215.1 (42.1) & 214.0 (42.0) & 217.6 (41.7) & 0.60 \\
 Gender (male) & 341 (51.9) & 64 (48.1) & 117 (53.9) & 160 (52.1) & 0.57\\
 Ever smoke (yes) & 379 (57.5) & 81 (60.9) & 126 (58.1) & 172 (56.0) & 0.63\\
 Hypertension medication (yes) & 445 (67.7) & 132 (99.2) & 196 (90.3) & 117 (38.1) & $<$0.001\\
 Diabetes (yes) & 90 (13.7) & 15 (11.3) & 27 (12.4) & 48 (15.6) & 0.38\\
   \hline
   \multicolumn{6}{l}{\textsuperscript{$\dagger$}\footnotesize{mean (sd) for continuous variables and frequency (percentage) for categorical variables.}}\\
   \multicolumn{6}{l}{\textsuperscript{*}\footnotesize{Comparing three SBP groups; ANOVA test for continuous variables and $\chi^2$ test for categorical variables.}}\\
\end{tabular}
}
\end{table}
% \end{sidewaystable}
% \end{landscape}

% \restoregeometry
% \pagestyle{plain}





\newpage
\subsubsection{\textsf{JAGS} Model File}
\textsf{JAGS} model file to fit QRJM of longitudinal and recurrent event data in simulation study.
{\scriptsize
\begin{verbatim}
model{
      k1 <- (1 - 2 * qt) / (qt * (1 - qt))
      k2 <- 2 / (qt * (1 - qt))

      # prior of random effects
      for (i in 1:I){ # I: unique subject id
        # prior for random effects
          u[i] ~ dnorm(0, tau)
      } # end of loop i

      # longitudinal process, BQR mixed model using ALD representation
      for (j in 1:N_l){ # N_l: number of longitudinal observations
          er[j] ~ dexp(sigma)
          mu[j] <- beta1 * X1_l[j] + beta2[X2_l[j]] + beta3 * t[j] + u[id_l[j]]+ k1 * er[j]
          prec[j] <- sigma / (k2 * er[j])
          y[j] ~ dnorm(mu[j], prec[j])
      } #end of j loop

      # recurrent events part, baseline hazard is set to constant c
      for(k in 1:I){
        for (l in (s[k]+1):s[k+1]){
          m1[l] <- beta1 * X1[k] + beta2[X2[k]] + beta3 * Ri1[l] + u[id_r[l]]
          m2[l] <- beta1 * X1[k] + beta2[X2[k]] + beta3 * Ri2[l] + u[id_r[l]]
          res[l] <- (exp(gamma * W[k] + alpha * m2[l]) - exp(gamma * W[k] + alpha * m1[l])) / (alpha * beta3)
          S[l] <- exp(- c * res[l])
          risk[l] <- c * exp(gamma * W[k] + alpha * m2[l])
          L[l] <- pow(risk[l], event[l]) * S[l] / 1E+08
          zeros[l] ~ dpois(-log(L[l]))
        } # end of l loop
      }#end of k loop

      # priors for other parameters
      alpha ~ dnorm(0, 0.001)
      beta1 ~ dnorm(0, 0.001)
      beta2[1] <- 0
      beta2[2] ~ dnorm(0, 0.001)
      beta3 ~ dnorm(0, 0.001)
      gamma ~ dnorm(0, 0.001)
      sigma ~ dgamma(0.001, 0.001)
      c ~ dunif(0.01, 10)
      tau <- pow(var, -2)
      var ~ dunif(0, 1000)
  }
\end{verbatim}
}