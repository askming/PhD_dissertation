\title{Bayesian Quantile Regression Joint Models: Inference And Dynamic Predictions}

\author{}
\date{}
\maketitle


\begin{center}
{\bf Abstract}
\end{center}

In the traditional joint models (JM) of a longitudinal and time-to-event outcome, a linear mixed model (LMM) assuming normal random errors is typically used to model the longitudinal process. However, in many circumstances, the normality assumption cannot be satisfied and the LMM is not an appropriate sub-model in the JM. In addition, as the LMM models the conditional mean of the longitudinal outcome, it is not appropriate if clinical interest lies in making inference or prediction on median, lower, or upper ends of the longitudinal process. To this end, quantile regression (QR) provides a flexible, distribution-free way to study covariate effects at different quantiles of the longitudinal outcome and it is robust not only to deviation from normality, but also to outlying observations. In this article, we present and advocate the linear quantile mixed model (LQMM) for the longitudinal process in the JM framework. Our development is motivated by a large prospective study of Huntington's Disease (HD) where primary clinical interest is in utilizing longitudinal motor scores and other early covariates to predict the risk of developing HD. We develop a Bayesian method based on the location-scale representation of the asymmetric Laplace distribution, assess its performance through an extensive simulation study, and demonstrate how this LQMM-based JM approach can be used for making subject-specific dynamic predictions of survival probability.

{\bf Key words:} Asymmetric Laplace distribution; Bayesian; Dynamic predictions; Huntington's disease; Joint models; Linear quantile mixed model.